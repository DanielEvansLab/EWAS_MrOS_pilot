%% BioMed_Central_Tex_Template_v1.06
%%                                      %
%  bmc_article.tex            ver: 1.06 %
%                                       %

%%IMPORTANT: do not delete the first line of this template
%%It must be present to enable the BMC Submission system to
%%recognise this template!!

%%%%%%%%%%%%%%%%%%%%%%%%%%%%%%%%%%%%%%%%%
%%                                     %%
%%  LaTeX template for BioMed Central  %%
%%     journal article submissions     %%
%%                                     %%
%%          <8 June 2012>              %%
%%                                     %%
%%                                     %%
%%%%%%%%%%%%%%%%%%%%%%%%%%%%%%%%%%%%%%%%%

%%%%%%%%%%%%%%%%%%%%%%%%%%%%%%%%%%%%%%%%%%%%%%%%%%%%%%%%%%%%%%%%%%%%%
%%                                                                 %%
%% For instructions on how to fill out this Tex template           %%
%% document please refer to Readme.html and the instructions for   %%
%% authors page on the biomed central website                      %%
%% https://www.biomedcentral.com/getpublished                      %%
%%                                                                 %%
%% Please do not use \input{...} to include other tex files.       %%
%% Submit your LaTeX manuscript as one .tex document.              %%
%%                                                                 %%
%% All additional figures and files should be attached             %%
%% separately and not embedded in the \TeX\ document itself.       %%
%%                                                                 %%
%% BioMed Central currently use the MikTex distribution of         %%
%% TeX for Windows) of TeX and LaTeX.  This is available from      %%
%% https://miktex.org/                                             %%
%%                                                                 %%
%%%%%%%%%%%%%%%%%%%%%%%%%%%%%%%%%%%%%%%%%%%%%%%%%%%%%%%%%%%%%%%%%%%%%

%%% additional documentclass options:
%  [doublespacing]
%  [linenumbers]   - put the line numbers on margins

%%% loading packages, author definitions

%\documentclass[twocolumn]{bmcart}% uncomment this for twocolumn layout and comment line below
\documentclass{bmcart}\usepackage[]{graphicx}\usepackage[]{color}
% maxwidth is the original width if it is less than linewidth
% otherwise use linewidth (to make sure the graphics do not exceed the margin)
\makeatletter
\def\maxwidth{ %
  \ifdim\Gin@nat@width>\linewidth
    \linewidth
  \else
    \Gin@nat@width
  \fi
}
\makeatother

\definecolor{fgcolor}{rgb}{0.345, 0.345, 0.345}
\newcommand{\hlnum}[1]{\textcolor[rgb]{0.686,0.059,0.569}{#1}}%
\newcommand{\hlstr}[1]{\textcolor[rgb]{0.192,0.494,0.8}{#1}}%
\newcommand{\hlcom}[1]{\textcolor[rgb]{0.678,0.584,0.686}{\textit{#1}}}%
\newcommand{\hlopt}[1]{\textcolor[rgb]{0,0,0}{#1}}%
\newcommand{\hlstd}[1]{\textcolor[rgb]{0.345,0.345,0.345}{#1}}%
\newcommand{\hlkwa}[1]{\textcolor[rgb]{0.161,0.373,0.58}{\textbf{#1}}}%
\newcommand{\hlkwb}[1]{\textcolor[rgb]{0.69,0.353,0.396}{#1}}%
\newcommand{\hlkwc}[1]{\textcolor[rgb]{0.333,0.667,0.333}{#1}}%
\newcommand{\hlkwd}[1]{\textcolor[rgb]{0.737,0.353,0.396}{\textbf{#1}}}%
\let\hlipl\hlkwb

\usepackage{framed}
\makeatletter
\newenvironment{kframe}{%
 \def\at@end@of@kframe{}%
 \ifinner\ifhmode%
  \def\at@end@of@kframe{\end{minipage}}%
  \begin{minipage}{\columnwidth}%
 \fi\fi%
 \def\FrameCommand##1{\hskip\@totalleftmargin \hskip-\fboxsep
 \colorbox{shadecolor}{##1}\hskip-\fboxsep
     % There is no \\@totalrightmargin, so:
     \hskip-\linewidth \hskip-\@totalleftmargin \hskip\columnwidth}%
 \MakeFramed {\advance\hsize-\width
   \@totalleftmargin\z@ \linewidth\hsize
   \@setminipage}}%
 {\par\unskip\endMakeFramed%
 \at@end@of@kframe}
\makeatother

\definecolor{shadecolor}{rgb}{.97, .97, .97}
\definecolor{messagecolor}{rgb}{0, 0, 0}
\definecolor{warningcolor}{rgb}{1, 0, 1}
\definecolor{errorcolor}{rgb}{1, 0, 0}
\newenvironment{knitrout}{}{} % an empty environment to be redefined in TeX

\usepackage{alltt}

%%% Load packages
\usepackage{amsthm,amsmath}
%\RequirePackage[numbers]{natbib}
%\RequirePackage[authoryear]{natbib}% uncomment this for author-year bibliography
%\RequirePackage{hyperref}
\usepackage[utf8]{inputenc} %unicode support
%\usepackage[applemac]{inputenc} %applemac support if unicode package fails
%\usepackage[latin1]{inputenc} %UNIX support if unicode package fails

%%%%%%%%%%%%%%%%%%%%%%%%%%%%%%%%%%%%%%%%%%%%%%%%%
%%                                             %%
%%  If you wish to display your graphics for   %%
%%  your own use using includegraphic or       %%
%%  includegraphics, then comment out the      %%
%%  following two lines of code.               %%
%%  NB: These line *must* be included when     %%
%%  submitting to BMC.                         %%
%%  All figure files must be submitted as      %%
%%  separate graphics through the BMC          %%
%%  submission process, not included in the    %%
%%  submitted article.                         %%
%%                                             %%
%%%%%%%%%%%%%%%%%%%%%%%%%%%%%%%%%%%%%%%%%%%%%%%%%

\def\includegraphic{}
\def\includegraphics{}

%%% Put your definitions there:
\startlocaldefs
\endlocaldefs

%%% Begin ...
\IfFileExists{upquote.sty}{\usepackage{upquote}}{}
\begin{document}

%%% Start of article front matter
\begin{frontmatter}

\begin{fmbox}
\dochead{Research}

%%%%%%%%%%%%%%%%%%%%%%%%%%%%%%%%%%%%%%%%%%%%%%
%%                                          %%
%% Enter the title of your article here     %%
%%                                          %%
%%%%%%%%%%%%%%%%%%%%%%%%%%%%%%%%%%%%%%%%%%%%%%

\title{DNA methylation associations with markers of inflammation in elderly men}

%%%%%%%%%%%%%%%%%%%%%%%%%%%%%%%%%%%%%%%%%%%%%%
%%                                          %%
%% Enter the authors here                   %%
%%                                          %%
%% Specify information, if available,       %%
%% in the form:                             %%
%%   <key>={<id1>,<id2>}                    %%
%%   <key>=                                 %%
%% Comment or delete the keys which are     %%
%% not used. Repeat \author command as much %%
%% as required.                             %%
%%                                          %%
%%%%%%%%%%%%%%%%%%%%%%%%%%%%%%%%%%%%%%%%%%%%%%

\author[
  addressref={cpmcri},                   % id's of addresses, e.g. {aff1,aff2}
%  corref={cpmcri},                       % id of corresponding address, if any
% noteref={n1},                        % id's of article notes, if any
  email={Daniel.Evans@ucsf.edu}   % email address
]{\inits{D.E.}\fnm{Daniel S.} \snm{Evans}}
\author[
  addressref={cpmcri},
  email={theresa.mau@ucsf.edu}
]{\inits{T.M.}\fnm{Theresa} \snm{Mau}}
\author[
  addressref={ucd},
  email={nelane@ucdavis.edu}
]{\inits{N.E.L.}\fnm{Nancy E.} \snm{Lane}}

%%%%%%%%%%%%%%%%%%%%%%%%%%%%%%%%%%%%%%%%%%%%%%
%%                                          %%
%% Enter the authors' addresses here        %%
%%                                          %%
%% Repeat \address commands as much as      %%
%% required.                                %%
%%                                          %%
%%%%%%%%%%%%%%%%%%%%%%%%%%%%%%%%%%%%%%%%%%%%%%

\address[id=cpmcri]{%                           % unique id
  \orgdiv{Research Institute},             % department, if any
  \orgname{California Pacific Medical Center},          % university, etc
  \city{San Francisco},                              % city
  \cny{CA}                                    % country
}
\address[id=ucd]{%
  \orgdiv{Department of Medicine},
  \orgname{University of California at Davis},
  %\street{},
  %\postcode{}
  \city{Davis},
  \cny{CA}
}

%%%%%%%%%%%%%%%%%%%%%%%%%%%%%%%%%%%%%%%%%%%%%%
%%                                          %%
%% Enter short notes here                   %%
%%                                          %%
%% Short notes will be after addresses      %%
%% on first page.                           %%
%%                                          %%
%%%%%%%%%%%%%%%%%%%%%%%%%%%%%%%%%%%%%%%%%%%%%%

%\begin{artnotes}
%%\note{Sample of title note}     % note to the article
%\note[id=n1]{Equal contributor} % note, connected to author
%\end{artnotes}

\end{fmbox}% comment this for two column layout

%%%%%%%%%%%%%%%%%%%%%%%%%%%%%%%%%%%%%%%%%%%%%%%
%%                                           %%
%% The Abstract begins here                  %%
%%                                           %%
%% Please refer to the Instructions for      %%
%% authors on https://www.biomedcentral.com/ %%
%% and include the section headings          %%
%% accordingly for your article type.        %%
%%                                           %%
%%%%%%%%%%%%%%%%%%%%%%%%%%%%%%%%%%%%%%%%%%%%%%%

\begin{abstractbox}

\begin{abstract} % abstract
\parttitle{Background} 
Epigenetic alterations are one of the hallmarks of aging, as age-related changes have been observed with various epigenetic markers, such as DNA methylation and post-translational modification of histones. Epigenome-wide association studies (EWAS) take an unbiased approach and test each DNA methylation marker for trait association. EWAS have found that specific DNA methylation sites in blood are associated with markers of inflammation, namely, C-reactive protein (CRP). CRP is also associated with multiple age-related conditions and diseases; thus, identifying molecular associations with CRP could help elucidate mechanisms important for aging. In this study, we report an EWAS of multiple markers of inflammation in elderly men and we examine whether biological aging defined by DNA methylation is associated with inflammation markers. 

\parttitle{Methods} 
DNA methylation was assayed using the Illumina Infinium MethylationEpic array using DNA isolated from whole blood. 

\parttitle{Results}
Text
\parttitle{Conclusions}
\end{abstract}

%%%%%%%%%%%%%%%%%%%%%%%%%%%%%%%%%%%%%%%%%%%%%%
%%                                          %%
%% The keywords begin here                  %%
%%                                          %%
%% Put each keyword in separate \kwd{}.     %%
%%                                          %%
%%%%%%%%%%%%%%%%%%%%%%%%%%%%%%%%%%%%%%%%%%%%%%

\begin{keyword}
\kwd{EWAS}
\kwd{CRP}
\end{keyword}

% MSC classifications codes, if any
%\begin{keyword}[class=AMS]
%\kwd[Primary ]{}
%\kwd{}
%\kwd[; secondary ]{}
%\end{keyword}

\end{abstractbox}
%
%\end{fmbox}% uncomment this for two column layout

\end{frontmatter}

%%%%%%%%%%%%%%%%%%%%%%%%%%%%%%%%%%%%%%%%%%%%%%%%
%%                                            %%
%% The Main Body begins here                  %%
%%                                            %%
%% Please refer to the instructions for       %%
%% authors on:                                %%
%% https://www.biomedcentral.com/getpublished %%
%% and include the section headings           %%
%% accordingly for your article type.         %%
%%                                            %%
%% See the Results and Discussion section     %%
%% for details on how to create sub-sections  %%
%%                                            %%
%% use \cite{...} to cite references          %%
%%  \cite{koon} and                           %%
%%  \cite{oreg,khar,zvai,xjon,schn,pond}      %%
%%                                            %%
%%%%%%%%%%%%%%%%%%%%%%%%%%%%%%%%%%%%%%%%%%%%%%%%

%%%%%%%%%%%%%%%%%%%%%%%%% start of article main body
% <put your article body there>

%%%%%%%%%%%%%%%%
%% Background %%
%%
\section*{Content}
Text and results for this section, as per the individual journal's instructions for authors.

\section*{Section title}
Text for this section\ldots
\subsection*{Sub-heading for section}
Text for this sub-heading\ldots
\subsubsection*{Sub-sub heading for section}
Text for this sub-sub-heading\ldots
\paragraph*{Sub-sub-sub heading for section}
Text for this sub-sub-sub-heading\ldots

In this section we examine the growth rate of the mean of $Z_0$, $Z_1$ and $Z_2$. In
addition, we examine a common modeling assumption and note the
importance of considering the tails of the extinction time $T_x$ in
studies of escape dynamics.
We will first consider the expected resistant population at $vT_x$ for
some $v>0$, (and temporarily assume $\alpha=0$)
%
\[
E \bigl[Z_1(vT_x) \bigr]=
\int_0^{v\wedge
1}Z_0(uT_x)
\exp (\lambda_1)\,du .
\]
%
If we assume that sensitive cells follow a deterministic decay
$Z_0(t)=xe^{\lambda_0 t}$ and approximate their extinction time as
$T_x\approx-\frac{1}{\lambda_0}\log x$, then we can heuristically
estimate the expected value as
%
\begin{equation}\label{eqexpmuts}
\begin{aligned}[b]
&      E\bigl[Z_1(vT_x)\bigr]\\
&\quad      = \frac{\mu}{r}\log x
\int_0^{v\wedge1}x^{1-u}x^{({\lambda_1}/{r})(v-u)}\,du .
\end{aligned}
\end{equation}
%
Thus we observe that this expected value is finite for all $v>0$ (also see \cite{koon,xjon,marg,schn,koha,issnic}).



\begin{knitrout}
\definecolor{shadecolor}{rgb}{0.969, 0.969, 0.969}\color{fgcolor}\begin{kframe}
\begin{alltt}
\hlstd{eset_Mvals} \hlkwb{<-} \hlkwd{read_rds}\hlstd{(}\hlstr{"../data/formatted/eset_Mvals_clean.rds"}\hlstd{)}
\hlstd{f_dat} \hlkwb{<-} \hlkwd{fData}\hlstd{(eset_Mvals)}
\hlkwd{sum}\hlstd{(f_dat}\hlopt{$}\hlstd{CVprobe} \hlopt{>=} \hlnum{100}\hlstd{)}
\end{alltt}
\begin{verbatim}
## [1] 5532
\end{verbatim}
\begin{alltt}
\hlkwd{sum}\hlstd{(f_dat}\hlopt{$}\hlstd{CVprobe} \hlopt{<} \hlnum{100}\hlstd{)}
\end{alltt}
\begin{verbatim}
## [1] 684481
\end{verbatim}
\begin{alltt}
\hlstd{eset_Mvals} \hlkwb{<-} \hlstd{eset_Mvals[f_dat}\hlopt{$}\hlstd{CVprobe} \hlopt{<} \hlnum{100}\hlstd{,]}
\hlstd{core_vars} \hlkwb{<-} \hlkwd{c}\hlstd{(}\hlstr{"ID"}\hlstd{,} \hlstr{"SITE"}\hlstd{,} \hlstr{"V3AGE1"}\hlstd{)}
\hlstd{outcome_var} \hlkwb{<-} \hlkwd{c}\hlstd{(}\hlstr{"CYCRPJH"}\hlstd{,} \hlstr{"CYTNFR2JH"}\hlstd{,} \hlstr{"CYIFNGJH"}\hlstd{,} \hlstr{"CYIL1BJH"}\hlstd{,} \hlstr{"CYIL6JH"}\hlstd{,} \hlstr{"CYTNFJH"}\hlstd{)}

\hlstd{p_dat1} \hlkwb{<-} \hlkwd{pData}\hlstd{(eset_Mvals)}
\hlstd{pheno} \hlkwb{<-} \hlkwd{read_csv}\hlstd{(}\hlstr{"../data/pheno/INFLAME.CSV"}\hlstd{)} \hlopt
        \hlkwd{select}\hlstd{(ID, SITE, CYCRPJH, LALYMP)}
\end{alltt}


{\ttfamily\noindent\itshape\color{messagecolor}{\#\# \\\#\# -- Column specification ----------------------------------------------------------------------------\\\#\# cols(\\\#\# \ \ .default = col\_double(),\\\#\# \ \ SITE = col\_character(),\\\#\# \ \ ID = col\_character()\\\#\# )\\\#\# i Use `spec()` for the full column specifications.}}\begin{alltt}
\hlstd{phenoV3} \hlkwb{<-} \hlkwd{read_csv}\hlstd{(}\hlstr{"../data/pheno/v3feb21.csv"}\hlstd{,} \hlkwc{guess_max} \hlstd{=} \hlnum{4682}\hlstd{)}
\end{alltt}


{\ttfamily\noindent\itshape\color{messagecolor}{\#\# \\\#\# -- Column specification ----------------------------------------------------------------------------\\\#\# cols(\\\#\# \ \ .default = col\_double(),\\\#\# \ \ ID = col\_character(),\\\#\# \ \ V3DATE = col\_character(),\\\#\# \ \ V3HVDATE = col\_character(),\\\#\# \ \ SITE = col\_character(),\\\#\# \ \ GISEDOD = col\_character(),\\\#\# \ \ CISTAFF = col\_character(),\\\#\# \ \ LSSTAFF = col\_character(),\\\#\# \ \ TMSTAFF = col\_character(),\\\#\# \ \ TBSTAFF = col\_character(),\\\#\# \ \ HWSTAFF = col\_character(),\\\#\# \ \ GSSTAFF = col\_character(),\\\#\# \ \ NFCSTAFF = col\_character(),\\\#\# \ \ NFWSTAFF = col\_character(),\\\#\# \ \ BPAASTAF = col\_character(),\\\#\# \ \ NPSTAFF = col\_character(),\\\#\# \ \ BPSTAFF = col\_character(),\\\#\# \ \ SCSTAFF = col\_character(),\\\#\# \ \ SCUSTAFF = col\_character(),\\\#\# \ \ V3AMSTF = col\_character(),\\\#\# \ \ TMTIMEM = col\_time(format = "{}"{})\\\#\# \ \ \# ... with 1 more columns\\\#\# )\\\#\# i Use `spec()` for the full column specifications.}}\begin{alltt}
\hlstd{phenoV3} \hlkwb{<-} \hlstd{phenoV3} \hlopt
        \hlkwd{select}\hlstd{(ID, HWBMI, TURSMOKE, V3AGE1)}
\hlstd{pheno2} \hlkwb{<-} \hlkwd{inner_join}\hlstd{(pheno, phenoV3,} \hlkwc{by} \hlstd{=} \hlstr{"ID"}\hlstd{)} \hlopt
        \hlkwd{mutate}\hlstd{(}\hlkwc{SITE} \hlstd{=} \hlkwd{as.factor}\hlstd{(SITE),}
               \hlkwc{TURSMOKE} \hlstd{=} \hlkwd{factor}\hlstd{(}\hlkwd{as.character}\hlstd{(TURSMOKE),} \hlkwc{levels} \hlstd{=} \hlkwd{c}\hlstd{(}\hlstr{"0"}\hlstd{,} \hlstr{"1"}\hlstd{,} \hlstr{"2"}\hlstd{),}
                                 \hlkwc{labels} \hlstd{=} \hlkwd{c}\hlstd{(}\hlstr{"never"}\hlstd{,} \hlstr{"former"}\hlstd{,} \hlstr{"current"}\hlstd{))}
               \hlstd{)}

\hlcom{#Work on CRP right now. Make loop for other inflammatory markers later.}
\hlstd{p_dat} \hlkwb{<-} \hlstd{p_dat1} \hlopt
  \hlkwd{left_join}\hlstd{(pheno2,} \hlkwd{c}\hlstd{(}\hlstr{"Sample_Name"} \hlstd{=} \hlstr{"ID"}\hlstd{))} \hlopt
  \hlkwd{arrange}\hlstd{(sampOrder)}
\hlkwd{map_int}\hlstd{(p_dat,} \hlkwa{function}\hlstd{(}\hlkwc{x}\hlstd{)} \hlkwd{sum}\hlstd{(}\hlkwd{is.na}\hlstd{(x)))}
\end{alltt}
\begin{verbatim}
##      Sample_Name      Sample_Well      Source_Well     Sample_Plate     Sample_Group          Pool_ID       Sentrix_ID Sentrix_Position         Basename        sampOrder        dupMrOSID             SITE 
##                0                0                0                0                0              150                0                0                0                0              143                7 
##          CYCRPJH           LALYMP            HWBMI         TURSMOKE           V3AGE1 
##                7                7                7                7                7
\end{verbatim}
\begin{alltt}
\hlcom{#Must remove missings from eset and pData}
\hlstd{mykeep} \hlkwb{<-} \hlopt{!}\hlkwd{is.na}\hlstd{(p_dat}\hlopt{$}\hlstd{SITE)} \hlopt{& !}\hlkwd{is.na}\hlstd{(p_dat}\hlopt{$}\hlstd{V3AGE1)} \hlopt{& !}\hlkwd{is.na}\hlstd{(p_dat}\hlopt{$}\hlstd{HWBMI)} \hlopt{& !}\hlkwd{is.na}\hlstd{(p_dat}\hlopt{$}\hlstd{TURSMOKE)}
\hlstd{eset_Mvals_mod} \hlkwb{<-} \hlstd{eset_Mvals[, mykeep]}
\hlstd{p_dat} \hlkwb{<-} \hlstd{p_dat[mykeep, ]}
\hlkwd{dim}\hlstd{(eset_Mvals_mod)}
\end{alltt}
\begin{verbatim}
## Features  Samples 
##   684481      143
\end{verbatim}
\begin{alltt}
\hlkwd{dim}\hlstd{(p_dat)}
\end{alltt}
\begin{verbatim}
## [1] 143  17
\end{verbatim}
\begin{alltt}
\hlcom{#Make table 1}
\hlstd{p_dat} \hlopt
        \hlkwd{summarize}\hlstd{(}\hlkwc{mean_CRP} \hlstd{=} \hlkwd{mean}\hlstd{(CYCRPJH),} \hlkwc{sd_CRP} \hlstd{=} \hlkwd{sd}\hlstd{(CYCRPJH))} \hlopt
        \hlkwd{kable}\hlstd{()}
\end{alltt}
\end{kframe}
\begin{tabular}{r|r}
\hline
mean\_CRP & sd\_CRP\\
\hline
3.010294 & 4.000611\\
\hline
\end{tabular}

\end{knitrout}


\section*{Appendix}
Text for this section\ldots

%%%%%%%%%%%%%%%%%%%%%%%%%%%%%%%%%%%%%%%%%%%%%%
%%                                          %%
%% Backmatter begins here                   %%
%%                                          %%
%%%%%%%%%%%%%%%%%%%%%%%%%%%%%%%%%%%%%%%%%%%%%%

\begin{backmatter}

\section*{Acknowledgements}%% if any
Text for this section\ldots

\section*{Funding}%% if any
Text for this section\ldots

\section*{Abbreviations}%% if any
Text for this section\ldots

\section*{Availability of data and materials}%% if any
Text for this section\ldots

\section*{Ethics approval and consent to participate}%% if any
Text for this section\ldots

\section*{Competing interests}
The authors declare that they have no competing interests.

\section*{Consent for publication}%% if any
Text for this section\ldots

\section*{Authors' contributions}
Text for this section \ldots

\section*{Authors' information}%% if any
Text for this section\ldots

%%%%%%%%%%%%%%%%%%%%%%%%%%%%%%%%%%%%%%%%%%%%%%%%%%%%%%%%%%%%%
%%                  The Bibliography                       %%
%%                                                         %%
%%  Bmc_mathpys.bst  will be used to                       %%
%%  create a .BBL file for submission.                     %%
%%  After submission of the .TEX file,                     %%
%%  you will be prompted to submit your .BBL file.         %%
%%                                                         %%
%%                                                         %%
%%  Note that the displayed Bibliography will not          %%
%%  necessarily be rendered by Latex exactly as specified  %%
%%  in the online Instructions for Authors.                %%
%%                                                         %%
%%%%%%%%%%%%%%%%%%%%%%%%%%%%%%%%%%%%%%%%%%%%%%%%%%%%%%%%%%%%%

% if your bibliography is in bibtex format, use those commands:
\bibliographystyle{bmc-mathphys} % Style BST file (bmc-mathphys, vancouver, spbasic).
\bibliography{bmc_article}      % Bibliography file (usually '*.bib' )
% for author-year bibliography (bmc-mathphys or spbasic)
% a) write to bib file (bmc-mathphys only)
% @settings{label, options="nameyear"}
% b) uncomment next line
%\nocite{label}

% or include bibliography directly:
% \begin{thebibliography}
% \bibitem{b1}
% \end{thebibliography}

%%%%%%%%%%%%%%%%%%%%%%%%%%%%%%%%%%%
%%                               %%
%% Figures                       %%
%%                               %%
%% NB: this is for captions and  %%
%% Titles. All graphics must be  %%
%% submitted separately and NOT  %%
%% included in the Tex document  %%
%%                               %%
%%%%%%%%%%%%%%%%%%%%%%%%%%%%%%%%%%%

%%
%% Do not use \listoffigures as most will included as separate files

\section*{Figures}
  \begin{figure}[h!]
  \caption{Sample figure title}
\end{figure}

\begin{figure}[h!]
  \caption{Sample figure title}
\end{figure}

%%%%%%%%%%%%%%%%%%%%%%%%%%%%%%%%%%%
%%                               %%
%% Tables                        %%
%%                               %%
%%%%%%%%%%%%%%%%%%%%%%%%%%%%%%%%%%%

%% Use of \listoftables is discouraged.
%%
\section*{Tables}
\begin{table}[h!]
\caption{Sample table title. This is where the description of the table should go}
  \begin{tabular}{cccc}
    \hline
    & B1  &B2   & B3\\ \hline
    A1 & 0.1 & 0.2 & 0.3\\
    A2 & ... & ..  & .\\
    A3 & ..  & .   & .\\ \hline
  \end{tabular}
\end{table}

%%%%%%%%%%%%%%%%%%%%%%%%%%%%%%%%%%%
%%                               %%
%% Additional Files              %%
%%                               %%
%%%%%%%%%%%%%%%%%%%%%%%%%%%%%%%%%%%

\section*{Additional Files}
  \subsection*{Additional file 1 --- Sample additional file title}
    Additional file descriptions text (including details of how to
    view the file, if it is in a non-standard format or the file extension).  This might
    refer to a multi-page table or a figure.

  \subsection*{Additional file 2 --- Sample additional file title}
    Additional file descriptions text.

\end{backmatter}
\end{document}
